GMCF's primary goal is to efficiently facilitate model coupling while reducing
the number of manual changes a model developer has to do. This goal is set to
improve on the current model coupling frameworks such as ESMF and OASIS.

A large number of existing models, the LES included, are written in a single
threaded fashion. The developers of these models aren't always well versed in
the computing science areas of multi-threading and shared- and
distributed-memory paralellism. This brings rise to a second goal of automatic
model parallelisation. This would significantly lower the barrier to using
multi-core multi-socket machines and clusters and dramatically improve the
performance of existing model code.

As a first stage in working towards automatic parallelisation, work was
conducted on GMCF to support intra-model communication. GMCF work had so far
concentrated on the communication between distinct models. This has led GMCF to
be the first model coupling framework to also support shared-memory
parallelisation. The parallelisation is currently manual however the
communication code using the same API as the model coupling communication code
so no further learning curve is required. This also means that when work towards
greater automation of model coupling is conducted, the intra-model
parallelisation work will also benefit from the changes.

In terms of parallelisation the LES itself using GMCF, the software engineering
effort for MPI LES made it easy to port from MPI to GMCF. The majority of the
communication code can be reused with the only changes being made were the
replacement of MPI API calls with GMCF API calls. This also shows how GMCF can
almost be used as a drop-in replacement for MPI meaning models that have already
been parallelised with MPI can either remain paralellised in this way when using
GMCF for model coupling, or can use GMCF for both parallelisation and coupling.
