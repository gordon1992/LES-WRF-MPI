GMCF is a relatively new framework with development beginning in 2014. GMCF's
primary function is for model coupling. One of long term goals of GMCF is to
automate as much of the model coupling process as possible. Current frameworks
currently require hand modification of existing model code and the writing of
additional code to describe and control the communication between models. This
can present a level of difficulty that a research team may not be willing to
accept so a framework that can automate a significant amount of the model
coupling work would be welcomed.

GMCF also aims to change the way model coupling is organised at runtime. The
current model coupling frameworks are focussed on clusters with single-core
machines and do not take into account the uptake in multicore and heterogeneous
nodes in modern clusters. The focus on single core nodes leads to a
communication that is necessarily distributed using MPI or equivalents and load
balancing between models is organised such that a model gets a fixed number of
nodes, proportional to its runtime compared with the runtimes of the other
models. With the advent of multicore clusters, a new load balancing technique
can be implemented, which is what GMCF does. GMCF aims to limit the coupling of
models to processes within a single node, with limited distributed memory
communication. The idea behind this is that communication in the cluster is more
symmetrical and each node can balance its load more evenly for improved
performance.

To date, GMCF has been used to couple the LES with WRF, acting as a proof of
concept to the new ideas behind the framework.
