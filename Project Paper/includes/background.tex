Climate scientists create independent simulations that each model a different
part of the planet's ecosystem. To model more complex aspects of the ecosystem,
and for more accurate results, multiple simulations are required to work
together.

There are three ways of doing having models work together \cite{Thevenin}. The
first is merging the model code bases together which generates a single
executable with all models having efficient data exchanges since all data is
contained within the one memory space. This solution isn't ideal since merging
code bases is difficult and further development of the individual models is now
more difficult.

The second is to use a communication library directly, such as MPI. Each model
can keep its own codebase with only explicit communication code being added to
each model. This solution also isn't ideal since the communication code isn't
generic since each model coupling will require its own communication code and
coupling more than two models together results in very complicated exchanges
being implemented.

The third solution is to use a model coupling library which abstracts the low
level communication complexities and acts as a ``middle man'' between each
model. For example, the LES has been coupled to the Weather Research and
Forecasting Model (WRF). This has led to more accurate wind velocities being
predicted compared with the predictions from WRF alone
\cite{Kinbara2010,Nakayama1998}, showing the benefit of having multiple models
interact with one another. There are many model coupling frameworks available,
including: the Model Coupling Toolkit (MCT) \cite{Jacob2005,Larson2005}; OASIS
\cite{Valcke2013,Valcke}; the Earth System Modeling Framework (ESMF)
\cite{Ramework2004}; and OpenPALM \cite{Piacentini2011}.

GMCF is a relatively new framework with development beginning in 2014. GMCF's
primary function is for model coupling. One of long term goals of GMCF is to
automate as much of the model coupling process as possible. Current frameworks
currently require hand modification of existing model code and the writing of
additional code to describe and control the communication between models.
