With the GMCF framework being still in the early stages of development, the
project itself has many items of future work in its roadmap including automating
as much of the model coupling tasks as possible. The future work discussed here
extends this by discussing potential performance enhancements and new features
that the GMCF framework could include.

\subsection{Performance changes}

With GMCF performance being competitive with MPI, but not completely on par or
improving on MPI, further performance investigations are required. Three
potential performance issues are thought to remain.

The first involves the C++ and Fortran boundary. This boundary is crossed many
times throughout an application's runtime. The boundary itself involves no
runtime overhead as the coupled application is a single binary however the
language changes limit the scope of what compiler optimisations can achieve. By
giving the Fortran API more capabilities such as handling packet sending and
receiving, making it more than just a thin wrapper, the compiler may be able to
optimise more heavily and thus improve performance.

The second performance improvement that should be investigated is a more direct
method of data transfer between threads in a shared-memory system. At the
moment, the sending thread creates an intermediate array to hold the data being
transferred, the framework creates a copy of the memory being transferred
between threads, and the receiving thread copies the data in its intermediate
array to the required array. By making changes to the API, it would be possible
to remove this additional intermediate memory copy and instead copy memory
directly between threads. This could also remove the need to send an
acknowledgement to the sending thread after the data has been copied to the
recipient, further reducing packet overhead.

The third performance improvement is an investigation into a tree-based
reduction algorithm such as the recursive doubling implemented by OpenMPI and
MPICH that has been shown to be highly scalable to large numbers of
nodes\cite{Thakur2003}. This will not provide a performance benefit for a single
node since the number of threads involved are too small but as distributed
memory capabilities are added to GMCF, and the framework is tested on hundreds
of CPU cores, the benefits of a tree-based reduction may become apparent.

\subsection{New capabilities}

A major new capability that is set for future work in the coming months is
distributed-memory parallelism. This is the primary difference between GMCF and
MPI as it stands. Enabling distributed-memory parallelism will allow GMCF model
coupling to use clusters and acts as a major step towards GMCF's novel model
coupling goal.

With the LES parallelisation conducted by a developer outwith the LES project,
the required paralellisation details such as shared arrays and halo sizes, had
to be inferred. As a result, it is feasible that the parallelisation can be made
automatic, even without annotations in some cases. An investigation into the
near-automatic parallelisation of existing single-threaded Fortran models should
be conducted as this will complement GMCFs goal of near-automated model
coupling.
