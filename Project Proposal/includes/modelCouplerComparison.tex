\begin{figure*}
    \begin{tabular}{|l|l|l|l|}
        \hline
        & ESMF & OpenPALM & OASIS3-MCT\\
        \hline
        Coupling Type & Static & Dynamic & Static\\
        \hline
        Coupling Behaviour & Sequential \& Concurrent & Sequential \& Concurrent & Sequential \& Concurrent\\
        \hline
        Single Executable & Supported & Supported & Supported\\
        \hline
        Multiple Executable & Limited Support & Supported & Supported\\
        \hline
        Local Parallelism & OpenMP \& MPI & OpenMP \& MPI & OpenMP \& MPI\\
        \hline
        Explicit Coupler Component & Yes & No & No\\
        \hline
    \end{tabular}
    \caption{Comparison of model coupling frameworks}
    \label{table:modelCouplerComparison}
\end{figure*}

Overall all frameworks are functionally similar however they employ different
approaches in how coupling is expressed and behaves.
Table~\ref{table:modelCouplerComparison} gives a feature set comparison.

As the table shows, OpenPALM is the only model coupling framework investigated
that supports dynamic coupling. With the LES and WRF both running at the same
time, the coupling would receive no benefit from being dynamic.

A coupled model system can be compiled to a single or multiple executable. With
the exception of ESMF, all model coupling frameworks offer full support for
multiple executables. Each model and, if exists, the coupler component, are
compiled to their own executables. The benefit of this is minor but very useful,
each model's build system can be used without significant modification (usually
just linking in the appropriate libraries). ESMF currently allow each component
(model) to be compiled to its own executable however the coupler component must
be linked into each executable for data exchange.

ESMF has a dedicated coupler component expressed in Fortran with explicit API
calls in each model, the coupler, and master thread for the coupling setup and
data transfer. OASIS3-MCT and OpenPALM use a plain text file to express the
coupling in a much simpler, condensed format however OpenPALM generates `glue
code' to create a master process whereas OASIS3-MCT requires no extra process.
OASIS3-MCT allows changes to be made to the coupling configuration without
requiring recompilation whereas OpenPALM may require recompilation due to
changes in the GUI-created `glue code' for the coupling.

\textbf{Architecture and Communication}

Each model coupler takes a different approach to the overall coupling
architecture and communication strategies.

ESMF has two layers, `superstructure' and `infrastructure' which sandwich the
user code being coupled. The superstructure is the upper layer which provides
high level components to the developer, such as Gridded Components to associate
with each model, Coupler Components to define how two models interact, and
States to encapsulate data between models. The infrastructure layer is the lower
layer which provides tools to manage the complexity of the different time steps
and data representations between models. The architecture is designed such that
each ESMF component doesn't need to know anything about the models it is coupled
with allowing component reuse and limited changes to the model code itself. The
design is also very flexible, allowing components to be coupled in any manner
required, for instance a hub-and-spokes or pair-wise coupling, and does not
restrict modes of execution to just solely sequential or solely concurrent.
Hub-and-spokes coupling allows a single coupler to have data exchanges between
all models whereas a pairwise coupling forces each coupler to be the
man-in-the-middle for exchanges between two models.

For communication, ESMF is also very flexible, even allowing model
communications to occur in the middle of a time step. There is no communication
mechanism hardwired into ESMF and, since all communication between models is
done via an explicit coupling component, any number of transformations can be
executed on data before being given to the receiving model. ESMF has a ``Uniform
communication API'' \cite{ESMF2014} meaning the same interface is used for
shared memory and distributed memory communication. This interface is realised
by the ESMF Virtual Machine, an abstraction of machine architecture, which
handles all of the resource management and communication methods. For load
balancing, ESMF supports load balancing packages such as Parmetis
\cite{Hoefler2010,Karypis1998} or developer-defined methods of load balancing.

OpenPALM has a simpler architecture to ESMF and a lot of the underlying
complexity is hidden by the supplied graphical user interface, PrePALM. There is
still an explicit coupler component however PrePALM generates the required glue
code and Makefile automatically. As a result, after compilation, a `palm\_main'
executable is created alongside the model executables. This is executed directly
and it manages the execution of the model executables as defined in the GUI.

For communication, OpenPALM uses an ``end point'' communication methodology
\cite{CERFACS2007} and components are kept independent from each other. This
means that when a model produces data that it would like to make available to a
component, it executes a non-blocking `PALM\_Put' call. If another model
requires data at some point, it executes a blocking `PALM\_Get' call. At the
model level, there is no need for the calls to match up with another model, it
is only when the coupling is being created are the get and put calls matched up.
The primary benefit of this is that if a model is being reused in another
coupling, it is unlikely that many of the calls themselves will need to be
modified, just the configuration in the GUI. Another benefit is that this also
means data transfer doesn't need to happen straight away, the recipient could be
executed later or may be in the middle of some computation so the put call is
non-blocking and the data is buffered, awaiting the corresponding get call.

OpenPALM works around the \textit{M-by-N problem} in a manner largely
transparent to the developer. For each model with multiple processes, OpenPALM
is aware of what process contains what portion of each array so, when processes
make get/put calls, it works out which processes need to communicate with each
other to complete the calls. The developer is not required to bundle together
the array to be sent nor does the library do this.

OASIS3-MCT's architecture is simpler still \cite{OASIS3-MCT2013}. OASIS3-MCT is
only a library just like ESMF and OpenPALM however, thanks to MCT, it requires
no explicit coupling components as with ESMF and OpenPALM. Any transformations
required for communication between models is executed on either the sender or
recipient processes since there is no central coupler executable. This also
necessitates point-to-point communication between models, similar to OpenPALM.
OASIS3-MCT's method of dealing with the \textit{M-by-N problem} is the same as
OpenPALM and `put' calls are also non-blocking with `get' calls blocking.
OASIS3-MCT also allows buffering of data through the use of coupler restart
files stored on disk during runtime.

\textbf{Coupling Overhead and Scaling}

In terms of coupling overhead caused by model coupling frameworks, there are few
pieces of literature discussing it in detail, especially for the model coupling
frameworks currently in use. For instance, Frickenhaus, Redler, and Post
\cite{StephanFrickenhausReneRedler2001} found an overhead of 10\% caused by the
MpCCI coupling library whereas \cite{Mogensen} found insignificant overhead in
their OASIS3 coupling. The OASIS team themselves found there to be insignificant
overhead in their toy example \cite{Valcke} however they also discuss an
un-cited example where a 6\% coupling overhead was observed. There were no
peer-reviewed articles discussing ESMF and OpenPALM overhead found.

In terms of scaling, similar problems exist with respect to peer-reviewed
articles. The respective model coupling frameworks show, for dummy examples,
that the framework scales well up to around 1000 processes, with increasing
overhead above that.

\textbf{Using the frameworks}

To get a feel for how each framework is used and behaves, a coupling was created
using all three frameworks.

A dummy coupling was created between two components in ESMF
\footnote{\url{https://github.com/gordon1992/LES-WRF-MPI/tree/master/
Examples/ESMF}}. The dummy coupling required around 350 source lines of code
just to set up a `no-op' model coupling. Due to the overly large amount of code
required to create a simple coupling, ESMF was not chosen as the preferred model
coupling framework to use.

For OpenPALM, the reliance on the GUI meant that a sampling coupling was not
created specifically for comparison between ESMF and OASIS3-MCT in terms of
required code changes, instead solely for getting a feel for the framework.
Overall, due to the seemingly heavyweight approach to creating a model coupling,
OpenPALM was not chosen as the preferred model coupling framework to use.

A dummy coupling was created between two components in OASIS3-MCT
\footnote{\url{https://github.com/gordon1992/LES-WRF-MPI/tree/master/
Examples/OASIS3-MCT}}. The coupling required an additional 100 sources lines of
code (including the plain text namcouple file) to create a coupling that
transferred a 2D array between two models.

Given OASIS3-MCTs simple namcouple file format and active development, it was
chosen as the preferred model coupling framework to use.
