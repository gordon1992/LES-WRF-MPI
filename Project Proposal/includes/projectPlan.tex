The project has a number of distinct stages with deliverables at the end of each
stage. There a number of major deliverables that have been identified. These
will be listed in the following section with the estimated delivery date in
bold. The estimated delivery dates correspond to an academic week.

\subsection{Deliverable Dates}

\begin{description}
    \item
    \item[Week 13] Implementation
    \begin{description}
        \item[Week 11] Distributed LES using MPI
        \item[Week 13] Coupled system of LES and WRF using OASIS3-MCT
    \end{description}
    \item[Week 27] Evaluation:
    \begin{description}
        \item[Week 18] MPI LES Performance using multiple MPI implementations
        \item[Week 20] LES WRF Coupling Performance
        \item[Week 27] Performance evaluation and qualitative comparison between
        the OASIS3-MCT coupling and the GMCF coupling
    \end{description}
    \item[Week 31] Final deliverable in the form of a paper and a presentation
    describing work completed.
\end{description}

Figure~\ref{fig:workPlan} displays this work plan as a Gantt chart.

\begin{figure*}
    \begin{ganttchart}[vgrid, y unit chart=0.5cm]{10}{31}
        \gantttitle{Academic Weeks}{22} \\
        \gantttitlelist{10,...,31}{1} \\
        \ganttgroup{Implementation}{10}{13} \\
        \ganttbar{Distributed LES}{10}{11} \\
        \ganttlinkedbar{LES WRF Coupling}{12}{13} \\
        \ganttmilestone{Implementation Complete}{13} \\
        \ganttgroup{Evaluation}{17}{27} \\
        \ganttbar{MPI LES Performance}{17}{18} \\
        \ganttlinkedbar{LES WRF Coupling Performance}{19}{20} \\
        \ganttlinkedbar{OASIS3-MCT and GMCF Comparison}{21}{27} \\
        \ganttmilestone{Evaluation Complete}{27} \\
        \ganttbar{Paper and presentation deliverable}{28}{31}
        \ganttlink{elem2}{elem3}
        \ganttlink{elem3}{elem5}
        \ganttlink{elem7}{elem8}
        \ganttlink{elem8}{elem9}
    \end{ganttchart}
    \caption{Gantt Chart for Work Plan}
    \label{fig:workPlan}
\end{figure*}

\subsection{Contingencies}

The project plan dates are based on a short feasibility study involving simple
MPI applications and dummy model couplings. If problems are encountered, due to
technical or temporal restrictions, alternative deliverables have been proposed.

The MPI LES Performance investigation would ideally span a range of node sizes
(up to 64 planned) and using different MPI implementations such as MPICH and
OpenMPI. Using multiple libraries could be too time consuming and, depending on
the implementations available, may be infeasible. Using one MPI implementation
would not be ideal however would still allow an in-depth review of the
scalability of a distribued LES to take place.

Given the lack of literature investigating the performance overheads of model
coupling in general and overhead of specific model couplers, an alternative
deliverable to the OASIS3-MCT and GMCF comparison would be to conduct a review
of the performance overhead of OASIS3-MCT. Since the current literature
\cite{StephanFrickenhausReneRedler2001, Mogensen} is at a disagreement between
model couplers having insignificant overhead to around 10\% overhead, this
alternative deliverable would be worthwhile work.
