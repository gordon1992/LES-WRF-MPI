
MPI is a recognised standard for sharing data between processes on parallel
systems by supporting message passing between nodes in a computer cluster or
supercomputer. The standard creates a common interface that allows any vendor to
create their own implementation and ensures that different implementations of
the same version of the standard can be interchanged without requiring user code
changes for the most part. This has led MPI to be viewed as the \textit{de
facto} standard for inter-process communication, especially across multiple
nodes. There are implementations of the standard for many languages including C,
Fortran, Java, and Python.

The latest standard is MPI-3
\footnote{\url{http://www.mpi-forum.org/docs/mpi-3.0/mpi30-report.pdf}} which is
mostly backwards-compatible with MPI-1 and MPI-2 however most vendors,
applications, and model coupling frameworks have MPI-1 or MPI-2 support so, for
the purpose of this project, MPI-3 specific enhancements will be ignored.

\subsubsection{Multithreaded MPI Communication}

In addition to computer clusters where there are multiple machines (nodes), each
individual node now has multiple CPU cores and sometimes even multiple CPU
sockets. This poses an additional problem for application and library
developers. Library developers have to think about how best to enable the use of
MPI inside a multithreaded application to prevent race conditions but still have
scalable performance. For application developers, multiple threads and MPI
individually bring their own complexity that has to be dealt with however there
is an additional challenge when considering how multiple threads make MPI calls.
For instance, it is possible for a multi-threaded process to make conflicting
MPI calls such as trying to receive data to the same location as another thread.
In addition to ensuring correctness, there are also performance concerns. A
summary of two papers investigating multi-threaded MPI performance is given
below.

D\'{o}zsa and Kuma et al.\ \cite{Kumar} use multichannel-enabled network
hardware, granular locking, atomic operations, and concurrency-aware message
queues to create a thread-safe MPI application with scalable performance. They
demonstrate that multithreaded applications can have a message passing rate that
scales with multiple threads. The authors send zero-byte messages to measure the
number of messages that can be sent per second as the number of threads grows.
There is a near-linear increase in performance as the number of threads is
increased whereas the default library implementation sees the message rate
\textit{decrease} as thread count increases. Performance is still worse than the
Deep Computing Message Framework (DCMF) which is a lower level messaging library
and has a message rate that scales almost linearly with the number of threads,
just like the MPI library, however DMCF has performance nearly seven times
greater than the optimised MPI implementation when both run on four threads. The
authors have not considered the latency of messages or raw throughput since
there may be an effect of the size of messages in a threaded environment and
zero-byte messages are unrealistic. The authors have only investigated message
rate for a small range of threads (1 to 4) however there can be single nodes
with 64 cores (16 core CPUs on a quad-socket motherboard) or more which is where
any multithreading-related performance problems would come to light.

Thakur and Gropp \cite{Thakur2009} create and use a number of performance
benchmarks using scenarios claimed to be close to typical applications. They
test the cost of thread safety, the ability for concurrent progress to be made,
and the possibilities for computation overlap. The authors' first test involves
a simple ping-pong latency test of a single threaded application that sets up
the MPI environment to allow multiple threads to make MPI calls without having
to explicitly synchronise with the other threads. The test shows negligible
latency on some MPI implementations. The validity of this test can be questioned
because, even though the MPI library is told to expect multiple threads making
calls simultaneously, this doesn't happen since the processes remain single
threaded. This means that the effect of any locks etc used by the MPI library
may not be being appropriately exercised to determine the overhead.

The authors' second test investigated cumulative bandwidth of MPICH2 and Open
MPI implementations, comparing each library's performance where parallelism is
achieved via processes or threads. For the Linux cluster the 1Gbit/s link
between nodes is saturated in all bar one case and thus no conclusions can be
made. For the Sun and IBM cases, threading is shown to dramatically decrease
throughput (over 50\%) in the threaded case compared to the process parallel
case. The bandwidth in Sun's multithreaded case was less than the saturated
bandwidth multithreaded case for the Linux cluster so this appears to show that
Sun can significantly improve the performance of their MPI implementation.

The other tests overall show how MPICH2 and Open MPI have fairly good
performance however the authors note that, due to the Gigabit Ethernet
interconnect, the overheads are generally masked since the Ethernet connection
is saturated in all cases, as shown by the cumulative bandwidth test. The paper
has a more comprehensive look into the performance differences of MPI libraries
and shows that significant work is required to minimise the overhead caused by
multithreaded MPI applications. The authors also make their benchmark test suite
available online for others to repeat their tests and see how different hardware
setups react and how later versions of MPI libraries improve.
