Individuals models for ocean, land, atmosphere, etc are useful in ther own right
however geoscientists are now wanting to model more complex scenarios that
require collaboration from multiple models. This collaboration can also lead to
more accurate results since additional effects from other models can be added to
the overall system. For example, coupling WRF and LES led to greater wind
velocity being predicted compared with running WRF on its own
\cite{Kinbara2010,Nakayama1998}.

Currently, WRF and LES are independent models and LES has two variants: the
original single threaded code and the OpenCL accelerated variant
\cite{Vanderbauwhede2014}. Both LES variants can only make use of the resources
of a single node. Vanderbauwhede and Takemi \cite{Vanderbauwhede2013} have also
investigated the benefits of GPU accelerating WRF and found this to be
``feasible and worthwhile'' however this is out of scope for this proposal.

The problem to be addressed is two-fold: creating a MPI variant of LES to allow
LES to run on multiple nodes in a computer cluster and then coupling this
variant of LES with WRF. The MPI variant of the LES will increase the
performance of the simulator since it will allow it to make use of a multi-node
distributed memory system such as a Beowulf cluster. The coupling of MPI LES
with WRF will create a system that benefits from scalable performance and high
accuracy results.
