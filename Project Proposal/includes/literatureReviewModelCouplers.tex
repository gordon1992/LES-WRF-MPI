Applications currently exist that model a single aspect of our Earth's
ecosystem. Each model is useful in its own right however more accurate and
detailed analysis can be conducted by making use of multiple models together.
This is not viable by using the models independently so there is significant
motivation to construct an efficient method of allowing applications to
interact.

There are three ways applications can be coupled together \cite{Thevenin}. The
first involves merging the two code bases. This solution is very efficient and
portable. A single process is created so data is shared via memory exchanges
which are lightweight however the process of merging code bases is complicated
and the two code bases are no longer independent making continued development of
each model difficult. Merging also causes problems for parallelism since each
model's parallel abilities are dictated by the overarching code that controls
both models. For these reasons this solution is not an appropriate way to couple
models.

The second solution \cite{Thevenin} is to use a communication library directly,
such as MPI. This allows each model to be kept independent with the only code
changes involved being related to the communication between models. The solution
still suffers from many problems. The coupling code is not generic since each
model needs to know exactly what model it is sending data to. Also, since the
MPI API is fairly low level, knowledge of parallel computing techniques is
required for efficient coupling. The problems of complexity and generality
increase if there are more than two models involved in the coupling with
complicated exchanges required. Using MPI for data exchanges is definitely a
viable way forward however a higher level solution is required to make coupling
many models viable.

The third solution \cite{Thevenin} follows on from the second. Instead of using
an MPI library directly, a coupling library is employed to abstract the low
level complexities of coupling away from the developer. A coupling library is a
separate piece of software that acts as the ``middle man'' between models and
facilitates data exchanges between models using a framework-specific interface.
Coupling libraries can also provide tools for interpolation such as re-gridding
\cite{Kuinghttons} if there are differences in data representation between
communicating models. Some coupling libraries also supply tools for performance
analysis with graphical representations of runtime execution characteristics.
There are many examples of coupling libraries in use, for example: the Model
Coupling Toolkit (MCT) \cite{Larson2005,Jacob2005}; OASIS3-MCT
\cite{Valcke,Valcke2013}; the Earth System Modeling Framework (ESMF)
\cite{Ramework2004} and OpenPALM \cite{Piacentini2011}.

There are two main types of coupling: static and dynamic
\cite{Thevenin,Piacentini2011}. In static coupling, all of the coupled
components have to start simultaneously at the beginning of the simulation. This
means any memory and CPU resources required by each model are allocated and
locked until the end of the simulation, even if a model does not start using the
resources until midway through the simulation. Dynamic coupling allows each
coupled component to start and end as required, freeing up resources when not in
use. Instead, resources are managed by the coupler. Generally there is a fixed
amount of resources available to the coupler at the beginning of the simulation
however the coupler is free to dynamically manage how the resources are used to
meet the requirements of the currently running components.

During runtime, the models themselves can either be run sequentially or
concurrently \cite{Maisonnave}. If two models need the results of each other
from a previous timestep, they can run concurrently. If one of the models needs
the result of the other from the current time step, the models run sequentially.
If more than two models are coupled together, both sequential and concurrent
execution may be required.

Coupling parallel models has a problem that is generic to all models and model
coupling frameworks. This is known as the \textit{Parallel Coupling Problem} or
the \textit{M-by-N problem}. Larson et al.\ \cite{Larson2005,Jacob2005} discuss
this in detail with respect to the creation of the Model Coupling Toolkit.
Consider a coupled system where a model is running on M processes and another
model is running on N processes. The \textit{M-by-N problem} is the problem of
transferring data between the M process model and the N process model. The
communication pattern required changes for different values for M and N. This
problem increases in complexity when more than two models are interacting with
each other. It is up to the model coupling framework to work out an efficient
and scalable solution for all possible communication patterns.

Each of the listed model couplers have many similarities however they also have
a number of differences in terms of how they couple models and the tools
offered. A brief overview of the model coupling frameworks is given below with a
comparison following afterwards.

\subsubsection{Model Couplers}

The Model Coupling Toolkit (MCT) \cite{Larson2005,Jacob2005} is a simple
coupling framework which has been very successful. It has been used to couple
many models, for example the Coupled Ocean Atmosphere Wave Sediment Transport
(COAWST) modelling system, COAMPS for coastal atmosphere simulation, and WRF.
The toolkit also forms the underlying code for the coupler in the Community
Climate System Model (CCSM). The project is suffering from bitrot, with the last
commit to the MCT repository in December 2012. However, the OASIS team have
integrated MCT code with their latest framework, OASIS3-MCT, so the framework is
still in active use. For these reasons MCT will not be considered in its own
right, instead considered with respect to OASIS3-MCT which is in active
development. The remaining three coupling frameworks have been looked at in
detail.

\textbf{ESMF} is in active development with an extensive API. The API is the
largest out of the three (the reference documentation is 1140 pages long, an
order of magnitude larger than the rest) and, since everything is done in
Fortran, the framework suffers from a significant amount of boilerplate code
that the user must write. The code is fairly straightforward so a tool could
probably be written to create most of the code automatically however such a tool
does not exist. The everything-in-code method also means recompilation is
required for any modification to the coupling configuration. On a positive note,
the extensive and granular API makes the framework arguably more flexible and
potentially easier to read since everything is in Fortran and explicitly
defined.

\textbf{OpenPALM} has not had a new release since January 2013, nearly two years
ago at the time of writing however seems to be in active use given that a
training session is scheduled for January 2015. OpenPALM uses a GUI, called
PrePALM, for defining the coupling configuration. There is an API with tens of
possible calls that need to be added to the models to be coupled. Each model
also requires specially defined Fortran comments that are `ID cards' that define
information on the model and what fields it has that can be sent to other
models. These comments are read by the GUI to help setup the coupling. The GUI
generates many files including a text file defining the coupling setup and a
number of glue code files to allow the coupling configuration to be compiled and
run. When compared with OASIS3-MCT this seems fairly complex and overall
heavyweight however the tool generates a Makefile so that the developer only
needs to interact with the GUI and doesn't need to look at the several files
created by the GUI.

OpenPALM's GUI also offers a feature to log and automatically analyse the
runtime of the coupled system. The GUI offers views on CPU time for each model,
a timeline of what model ran when, and even a `playback' feature where the flow
of execution can be viewed graphically. This can be very useful in analysing and
debugging performance issues.

\textbf{OASIS3-MCT} is in active development, with the latest release in May
2013. Also, from May 2014, WRF has added OASIS3-MCT coupling code in its version
3.6 release \cite{ENES2014}. This official coupling support makes using
OASIS3-MCT for the LES-WRF coupling a more attractive option since some of the
code changes required in WRF may already be completed.

The previous version of OASIS was OASIS3. The move from OASIS3 to OASIS3-MCT
brought a lot of changes due to the integration of MCT code, for instance the
`namcouple' file which contains the coupling configuration is much simpler with
a lot of content deprecated. This doesn't seem to reduce the feature set of the
coupling framework but does given the impression that the OASIS framework could
benefit from a clean up of documentation and the framework code base.

Depending on the coupling configuration, OASIS3-MCT may require the use of
`coupling restart files' which are netCDF \cite{Unidata} files containing data
in use. This is used to read data initially and for storing data to be sent to
another model if the recipient model doesn't need it yet. This reliance on I/O
operations is a potential performance concern which will need to be investigated
from runtime execution statistics. On the other hand, Polcher et al.\
\cite{Polcher2013} did not find I/O to be a bottleneck in their ORCHIDEE
coupling when executed on a cluster with up to 20 CPUs which is a promising
result.

The next version of OASIS3-MCT will bring a GUI for creating the namcouple file
automatically and a performance analysis tool, LUCIA \cite{Maisonnave}, which
may bring its feature set more inline with OpenPALM's offerings.

\subsubsection{Comparison of Model Couplers}
\begin{figure*}
    \begin{tabular}{|l|l|l|l|}
        \hline
        & ESMF & OpenPALM & OASIS3-MCT\\
        \hline
        Coupling Type & Static & Dynamic & Static\\
        \hline
        Coupling Behaviour & Sequential \& Concurrent & Sequential \& Concurrent & Sequential \& Concurrent\\
        \hline
        Single Executable & Supported & Supported & Supported\\
        \hline
        Multiple Executable & Limited Support & Supported & Supported\\
        \hline
        Local Parallelism & OpenMP \& MPI & OpenMP \& MPI & OpenMP \& MPI\\
        \hline
        Explicit Coupler Component & Yes & No & No\\
        \hline
    \end{tabular}
    \caption{Comparison of model coupling frameworks}
    \label{table:modelCouplerComparison}
\end{figure*}

Overall all frameworks are functionally similar however they employ different
approaches in how coupling is expressed and behaves.
Table~\ref{table:modelCouplerComparison} gives a feature set comparison.

As the table shows, OpenPALM is the only model coupling framework investigated
that supports dynamic coupling. With the LES and WRF both running at the same
time, the coupling would receive no benefit from being dynamic.

A coupled model system can be compiled to a single or multiple executable. With
the exception of ESMF, all model coupling frameworks offer full support for
multiple executables. Each model and, if exists, the coupler component, are
compiled to their own executables. The benefit of this is minor but very useful,
each model's build system can be used without significant modification (usually
just linking in the appropriate libraries). ESMF currently allow each component
(model) to be compiled to its own executable however the coupler component must
be linked into each executable for data exchange.

ESMF has a dedicated coupler component expressed in Fortran with explicit API
calls in each model, the coupler, and master thread for the coupling setup and
data transfer. OASIS3-MCT and OpenPALM use a plain text file to express the
coupling in a much simpler, condensed format however OpenPALM generates `glue
code' to create a master process whereas OASIS3-MCT requires no extra process.
OASIS3-MCT allows changes to be made to the coupling configuration without
requiring recompilation whereas OpenPALM may require recompilation due to
changes in the GUI-created `glue code' for the coupling.

\textbf{Architecture and Communication}

Each model coupler takes a different approach to the overall coupling
architecture and communication strategies.

ESMF has two layers, `superstructure' and `infrastructure' which sandwich the
user code being coupled. The superstructure is the upper layer which provides
high level components to the developer, such as Gridded Components to associate
with each model, Coupler Components to define how two models interact, and
States to encapsulate data between models. The infrastructure layer is the lower
layer which provides tools to manage the complexity of the different time steps
and data representations between models. The architecture is designed such that
each ESMF component doesn't need to know anything about the models it is coupled
with allowing component reuse and limited changes to the model code itself. The
design is also very flexible, allowing components to be coupled in any manner
required, for instance a hub-and-spokes or pair-wise coupling, and does not
restrict modes of execution to just solely sequential or solely concurrent.
Hub-and-spokes coupling allows a single coupler to have data exchanges between
all models whereas a pairwise coupling forces each coupler to be the
man-in-the-middle for exchanges between two models.

For communication, ESMF is also very flexible, even allowing model
communications to occur in the middle of a time step. There is no communication
mechanism hardwired into ESMF and, since all communication between models is
done via an explicit coupling component, any number of transformations can be
executed on data before being given to the receiving model. ESMF has a ``Uniform
communication API'' \cite{ESMF2014} meaning the same interface is used for
shared memory and distributed memory communication. This interface is realised
by the ESMF Virtual Machine, an abstraction of machine architecture, which
handles all of the resource management and communication methods. For load
balancing, ESMF supports load balancing packages such as Parmetis
\cite{Hoefler2010,Karypis1998} or developer-defined methods of load balancing.

OpenPALM has a simpler architecture to ESMF and a lot of the underlying
complexity is hidden by the supplied graphical user interface, PrePALM. There is
still an explicit coupler component however PrePALM generates the required glue
code and Makefile automatically. As a result, after compilation, a `palm\_main'
executable is created alongside the model executables. This is executed directly
and it manages the execution of the model executables as defined in the GUI.

For communication, OpenPALM uses an ``end point'' communication methodology
\cite{CERFACS2007} and components are kept independent from each other. This
means that when a model produces data that it would like to make available to a
component, it executes a non-blocking `PALM\_Put' call. If another model
requires data at some point, it executes a blocking `PALM\_Get' call. At the
model level, there is no need for the calls to match up with another model, it
is only when the coupling is being created are the get and put calls matched up.
The primary benefit of this is that if a model is being reused in another
coupling, it is unlikely that many of the calls themselves will need to be
modified, just the configuration in the GUI. Another benefit is that this also
means data transfer doesn't need to happen straight away, the recipient could be
executed later or may be in the middle of some computation so the put call is
non-blocking and the data is buffered, awaiting the corresponding get call.

OpenPALM works around the \textit{M-by-N problem} in a manner largely
transparent to the developer. For each model with multiple processes, OpenPALM
is aware of what process contains what portion of each array so, when processes
make get/put calls, it works out which processes need to communicate with each
other to complete the calls. The developer is not required to bundle together
the array to be sent nor does the library do this.

OASIS3-MCT's architecture is simpler still \cite{OASIS3-MCT2013}. OASIS3-MCT is
only a library just like ESMF and OpenPALM however, thanks to MCT, it requires
no explicit coupling components as with ESMF and OpenPALM. Any transformations
required for communication between models is executed on either the sender or
recipient processes since there is no central coupler executable. This also
necessitates point-to-point communication between models, similar to OpenPALM.
OASIS3-MCT's method of dealing with the \textit{M-by-N problem} is the same as
OpenPALM and `put' calls are also non-blocking with `get' calls blocking.
OASIS3-MCT also allows buffering of data through the use of coupler restart
files stored on disk during runtime.

\textbf{Coupling Overhead and Scaling}

In terms of coupling overhead caused by model coupling frameworks, there are few
pieces of literature discussing it in detail, especially for the model coupling
frameworks currently in use. For instance, Frickenhaus, Redler, and Post
\cite{StephanFrickenhausReneRedler2001} found an overhead of 10\% caused by the
MpCCI coupling library whereas \cite{Mogensen} found insignificant overhead in
their OASIS3 coupling. The OASIS team themselves found there to be insignificant
overhead in their toy example \cite{Valcke} however they also discuss an
un-cited example where a 6\% coupling overhead was observed. There were no
peer-reviewed articles discussing ESMF and OpenPALM overhead found.

In terms of scaling, similar problems exist with respect to peer-reviewed
articles. The respective model coupling frameworks show, for dummy examples,
that the framework scales well up to around 1000 processes, with increasing
overhead above that.

\textbf{Using the frameworks}

To get a feel for how each framework is used and behaves, a coupling was created
using all three frameworks.

A dummy coupling was created between two components in ESMF
\footnote{\url{https://github.com/gordon1992/LES-WRF-MPI/tree/master/
Examples/ESMF}}. The dummy coupling required around 350 source lines of code
just to set up a `no-op' model coupling. Due to the overly large amount of code
required to create a simple coupling, ESMF was not chosen as the preferred model
coupling framework to use.

For OpenPALM, the reliance on the GUI meant that a sampling coupling was not
created specifically for comparison between ESMF and OASIS3-MCT in terms of
required code changes, instead solely for getting a feel for the framework.
Overall, due to the seemingly heavyweight approach to creating a model coupling,
OpenPALM was not chosen as the preferred model coupling framework to use.

A dummy coupling was created between two components in OASIS3-MCT
\footnote{\url{https://github.com/gordon1992/LES-WRF-MPI/tree/master/
Examples/OASIS3-MCT}}. The coupling required an additional 100 sources lines of
code (including the plain text namcouple file) to create a coupling that
transferred a 2D array between two models.

Given OASIS3-MCTs simple namcouple file format and active development, it was
chosen as the preferred model coupling framework to use.

