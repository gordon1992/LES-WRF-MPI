Applications currently exist that model a single aspect of our Earth's
ecosystem. Each model is useful in their own right however more accurate and
detailed analysis can be conducted by making use of multiple models. This isn't
viable by using the models independently so there is significant motivation to
construct an efficient method of allowing applications to interact.

There are three ways applications can be coupled together \cite{Thevenin}. The
first involves merging the two code bases. This solution is very efficient and
portable as a single process is created. Data is shared via memory exchanges
which are lightweight however the process of merging is complicated and the two
code bases are no longer independent making continued development of each model
difficult. Merging also causes problems for parallelism since each model's
parallel abilities are dictated by the overarching code that controls both
models. For these reasons this solution is not an appropriate way to couple
models.

The second solution \cite{Thevenin} is to use a communication library directly,
such as MPI. This allows each model to be kept independent with the only code
changes involved being related to the communication between models. The solution
still suffers from many problems. The coupling code isn't generic since each
model needs to know exactly what model it is sending data to. Also, since the
MPI API is fairly low level, knowledge of parallel computing techniques is
required for efficient coupling. The problems of complexity and generality
increase if there are many models involved in the coupling with complicated
exchanges required. Using MPI for data exchanges is definitely a viable way
forward however a higher level solution is required to make coupling many models
viable.

The third solution \cite{Thevenin} follows on from the second. Instead of using
an MPI library directly, a coupling library is employed to abstract the low
level complexities of coupling away from the developer. A coupling library is a
separate piece of software that acts as the ``middle man'' between models and
facilitates data exchanges between models using a framework-specific interface.
Coupling libraries can also provide tools for interpolation such as regridding
\cite{Kuinghttons} if there are differences in data representation between
communicating models. Some coupling libraries also supply tools for performance
analysis with graphical representations of runtime execution characteristics.
There are many examples of coupling libraries in use, for example: the Model
Coupling Toolkit (MCT) \cite{Larson2005,Jacob2005}; OASIS3-MCT
\cite{Valcke,Valcke2013}; the Earth System Modeling Framework (ESMF)
\cite{Ramework2004} and OpenPALM \cite{Piacentini2011}.

There are two main types of coupling: static and dynamic
\cite{Thevenin,Piacentini2011}. In static coupling, all of the coupled
components have to start simultaneously at the beginning of the simulation. This
means any memory and CPU resources required by each model are allocated and
locked until the end of the simulation, even if a model doesn't start using the
resources until midway through the simulation. Dynamic coupling allows each
coupled component to start and end as required, freeing up resources when not in
use. Instead, resources are managed by the coupler. Generally there is a fixed
amount of resources available to the coupler at the beginning of the simulation
however the coupler is free to dynamically manage how the resources are used to
meet the requirements of the currently running components.

During runtime, the models themselves can either be run sequentially or
concurrently \cite{Maisonnave}. If two models need the results of each other
from a previous timestep, they can run concurrently. If one of the models needs
the result of the other from the current time step, the models run sequentially.
If more than two models are coupled together, both sequential and concurrent
execution may be required.

Coupling parallel models has a problem that is generic to all models and model
coupling frameworks. This is known as the \textit{Parallel Coupling Problem} or
the \textit{M-by-N problem}. Larson et al.\ \cite{Larson2005,Jacob2005} discuss
this in detail with respect to the creation of the Model Coupling Toolkit.
Consider a coupled system where a model is running on M processes and another
model is running on N processes. The \textit{M-by-N problem} is the problem of
transferring data between the M process model and the N process model. The
communication pattern required changes for different values for M and N. This
problem increases in complexity when more than two models are interacting with
each other. It is up to the model coupling framework to work out an efficient
and scalable solution for all possible communication patterns.

Each of the listed model couplers have many similarities however they also have
a number of differences in terms of how they couple models and the tools
offered. A comparison of model couplers is given below.

\subsubsection{Model Couplers}

The Model Coupling Toolkit (MCT) \cite{Larson2005,Jacob2005} is a simple
coupling framework which has been very successful. It has been used to couple
many models, for example the Coupled Ocean Atmosphere Wave Sediment Transport
(COAWST) modelling system, COAMPS for coastal atmosphere simulation, and WRF.
The toolkit also forms the underlying code for the coupler in the Community
Climate System Model (CCSM). The project is suffering from bitrot, with the last
commit to the MCT repository in December 2012. However, the OASIS team have
integrated MCT code with their latest framework, OASIS3-MCT, so the framework is
still in active use. For these reasons MCT will not be considered in its own
right, instead considered with respect to OASIS3-MCT which is in active
development.

The remaining three coupling frameworks have been looked at in detail. Overall
all frameworks are functionally similar however they employ different approaches
in how coupling is expressed and behaves. ESMF has a dedicated coupler component
expressed in Fortran with explicit API calls in each model, the coupler, and
master thread for the coupling setup and data transfer. OASIS3-MCT and OpenPALM
use a plain text file to express the coupling in a much simpler, condensed
format however OpenPALM generates `glue code' to create a master process whereas
OASIS3-MCT requires no extra process. OASIS3-MCT allows changes to be made to
the coupling configuration without requiring recompilation whereas OpenPALM may
require recompilation due to changes in the GUI-created `glue code' for the
coupling.

\textbf{ESMF} is in active development with an extensive API. The API is the
largest out of the three (the reference documentation is 1140 pages long, an
order of magnitude larger than the rest) and, since everything is done in
Fortran, the framework suffers from a significant amount of boilerplate code
that the user must write. The code is fairly straightforward so a tool could
probably be written to create most of the code automatically however such a tool
does not exist. The everything-in-code method also means recompilation is
required for any modification to the coupling configuration. On a positive note,
the extensive and granular API makes the framework arguably more flexible and
potentially easier to read since everything is in Fortran and explicitly
defined.

\textbf{OpenPALM} hasn't had a new release since January 2013, nearly two years
ago at the time of writing. OpenPALM uses a GUI, called PrePALM, for defining
the coupling configuration. There is an API with tens of possible calls that
need to be added to the models to be coupled. Each model also requires specially
defined Fortran comments that are `ID cards' that define information on the
model and what fields it has that can be sent to other models. These comments
are read by the GUI to help setup the coupling. The GUI generates many files
including a text file defining the coupling setup and a number of glue code
files to allow the coupling configuration to be compiled and run. When compared
with OASIS3-MCT this seems fairly complex and overall heavyweight however the
tool generates a Makefile so that the end user only needs to interact with the
GUI and doesn't need to look at the several files created by the GUI.

OpenPALM also offers a feature to log and automatically analyse the runtime of
the coupled system. The GUI offers views on CPU time for each model, a timeline
of what model ran when, and even a `playback' feature where the flow of
execution can be viewed graphically.

\textbf{OASIS3-MCT} is in active development, with the latest release in May
2013. Also, from May 2014, WRF has added OASIS3-MCT coupling code in its version
3.6 release \cite{ENES2014}. This official coupling support makes using
OASIS3-MCT for the LES-WRF coupling a more attractive option.

The previous version of OASIS was OASIS3. The move from OASIS3 to OASIS3-MCT
brought a lot of changes due to the integration of MCT code, for instance the
`namcouple' file which contains the coupling configuration is much simpler with
a lot of content deprecated. This doesn't seem to reduce the feature set of the
coupling framework but does given the impression that the OASIS framework could
benefit from a clean up of documentation and the framework code base.

Depending on the coupling configuration, OASIS3-MCT may require the use of
coupling `restart files' which are netCDF \cite{Unidata} files containing data
in use. This is used to read data initially and for storing data to be sent to
another model if the recipient model doesn't need it yet. This reliance on I/O
operations is a potential performance concern which will need to be investigated
during runtime. On the other hand, Polcher et al.\ \cite{Polcher2013} did not
find I/O to be a bottleneck in their ORCHIDEE coupling when executed on a
cluster with up to 20 CPUs.

The next version of OASIS3-MCT will bring a GUI for creating the namcouple file
automatically and a performance analysis tool, LUCIA \cite{Maisonnave}, which
may bring its feature set more inline with OpenPALM's offerings.
