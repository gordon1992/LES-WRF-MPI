\documentclass[a4page,twocolumn]{article}

\usepackage{fullpage}

\usepackage{titlesec}
\titlespacing{\section}{0pt}{0pt}{0pt}
\titlespacing{\subsection}{0pt}{0pt}{0pt}
\titlespacing{\subsubsection}{0pt}{0pt}{0pt}

\title{Coupling the distributed Large Eddy Simulation (LES) and Weather
Research and Forecasting model (WRF) using OASIS3-MCT and MPI}

\author{\large\textsc{Gordon Reid - 1002536r}\\University of Glasgow}

\date{\today}

\begin{document}
\setlength{\parskip}{0.3cm}

\maketitle

\begin{abstract}

\noindent Scientists studying the geosciences have created software which models
different aspects of our planet's ecosystem, for example ocean, wind, land and
atmosphere. On their own, each model is useful however for higher accuracy, and
to model more complex aspects of our climate, these models need to be coupled
together. The project proposed involves creating a distributed version of the
Large Eddy Simulator (LES) then coupling the LES with the Weather Research and
Forecasting model (WRF). The coupled system should scale to large computing
clusters. Ideally the coupling would also be generic such that the
OpenCL-accelerated LES could be used to allow greater performance on
heterogeneous systems.

\end{abstract}

\section*{Introduction}

Our planet's ecosystem is highly complex. Geoscientists that are interested in
modelling this ecosystem have created numerous systems which model one aspect of
it, for instance creating a Large Eddy Simulator (LES) to study turbulent air
flows \cite{Nakayama2011,Nakayama2012} and the Weather Research and Forecasting
Model (WRF) for mesoscale weather prediction. For the last decade or so,
interest has evolved from individual models to combining models
\cite{Michalakes2010}. With this trend, a number of model coupling frameworks
have been created to support the desire for co-simulation. A number of modeling
frameworks have been looked at, including: The Model Coupling Toolkit (MCT)
\cite{Larson2005}; OASIS3 \cite{Valcke,Valcke2013}; the Earth System Modeling
Framework (ESMF) \cite{Ramework2004}; and OpenPALM \cite{Piacentini2011}.

\section*{Statement of Problem}

\section*{Literature Review}

\section*{Proposed Approach}

\section*{Work Plan}

\section*{Summary}

\bibliographystyle{plain}
\bibliography{1002536r}

\end{document}
