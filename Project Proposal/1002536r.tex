\documentclass[twoside]{article}
\usepackage{multicol}
\usepackage{fullpage}
\usepackage{fancyhdr}
	\pagestyle{fancy}
	\fancyhead{}
	\fancyfoot{}
	\fancyhead[C]{Project Proposal $\bullet$ Level 5 Computing Science}
	\fancyfoot[RO,LE]{\thepage}
\usepackage{hyperref}
\title{Coupling the distributed Large Eddy Simulation (LES) and Weather
Research and Forecasting model (WRF) using OASIS3-MCT and MPI}
\author{
	\large
	\textsc{Gordon Reid - 1002536r}\\University of Glasgow
}
\date{\today}

\begin{document}

\maketitle
\thispagestyle{fancy}

\begin{abstract}

\noindent Weather and climate scientists have created software which models
different aspects of our planet's ecosystem, for example ocean, wind, land, and
atmosphere. On their own, each model is useful however for higher accuracy, and
to model more complex aspects of our climate, these models need to be coupled
together. The project proposed involves coupling the Large Eddy Simulator (LES)
with the Weather Research and Forecasting model (WRF) and having the coupled
system scale to large computing clusters. Ideally the coupling would also be
generic such that the OpenCL-accelerated LES could be used to allow greater
performance on heterogeneous systems.

\end{abstract}

\begin{multicols}{2}

\section*{Introduction}

\section*{Statement of Problem}

\section*{Literature Review}

\section*{Proposed Approach}

\section*{Work Plan}

\section*{Summary}

\bibliographystyle{plain}
\bibliography{1002536r}

\end{multicols}

\end{document}
